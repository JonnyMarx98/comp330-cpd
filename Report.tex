% Please do not change the document class
\documentclass{scrartcl}

% Please do not change these packages
\usepackage[hidelinks]{hyperref}
\usepackage[none]{hyphenat}
\usepackage{setspace}
\doublespace

% You may add additional packages here
\usepackage{amsmath}

% Please include a clear, concise, and descriptive title
\title{Continuing Personal Development Report}

% Please do not change the subtitle
\subtitle{COMP330 - CPD Report}

% Please put your student number in the author field
\author{1603748}

\begin{document}

\maketitle

\section*{Introduction}
This first term of my final year has been the most challenging year of the course so far, and I am expecting next term to be just as challenging or more so. I have felt a lot of pressure to perform well because the grades all count towards my final grade, however due to my poor time management I began to feel demotivated and stress as deadlines came closer. I believe this semester has been useful in preparing me for next semester and the future, I know what I need to do to improve and what to expect now. 

\section{Not asking for help when I need it}
I have always been fairly shy when it comes to asking for help, I only tend to ask people who I feel close to for help. I often get stuck while working on programming tasks and only ask my friends who are sitting next to me for assistance, I need to not be afraid to seek further help from a lecturer or another student. Because I rarely ask for help I tend to get stressed out because I can't solve my problem. This term I have struggled a lot on my dissertation project, and should have sought advice from my supervisor more often, I feel that I would have been less stressed if I had got some help when I needed it, instead of trying to figure it out alone. To overcome this obstacle I will attempt to not avoid asking for help when I need it, if I need the help I will ask for it. There is no other way to over way to improve this other than doing it as often as possible. 

\section{Specializing in a programming language}
This course covers a wide range of different languages, rather than focusing on one language and focusing on that for three years, this course doesn't focus on any one language and instead includes a large variety of languages. This means I don't spend a lot of time improving my skills in each language, especially the ones that I use for only one assignment, for example javascript for our website assignment. Thinking about my future career, I think it is important that I consider which programming languages I find most enjoyable and which ones are most valuable to employers. If I want to work for a AAA games development company I would need to practice C++ on a regular basis, however I am uncertain about if I want to work for a AAA company so I will instead practice my C# skills. I will practice this by developing a simple mobile game on unity, and if I have time, multiple mobile games. 

\section{Portfolio}
A portfolio is an extremely important thing to have in the games industry, it shows how much experience I have, and what type of projects I have been capable of developing. At the moment I have not put together a portfolio at all, I have quite a lot of projects that I could put on it though, for example my team games from each year and other projects I have made for assignments like my HoloLens game. I would like to gain more experience, other than the experience I will inevitably get from university projects.The first thing I plan on developing is a mobile game as I said in the previous section. I will also put together a website for my portfolio using github before the end of next semester. 

\section{Self discipline}
This semester my biggest challenge was completing my dissertation, I found it very tedious sometimes, and I definitely do not enjoy working on it as much as other things. The idea of reading a large amount of academic papers and writing 6000 words is extremely daunting to me, in turn this makes me feel demotivated to work on it because I expect it to be a tedious and unexciting task. This then inevitably leads to heaps of procrastination and eventually lots of stress. To overcome this I would like to become more self disciplined, I would like to have the ability to do tasks even it if I think I will not enjoy them. I think in order to become more disciplined I need to be motivated to do work, so I will from now on attempt to do the work that I favour the least first, and reward myself afterwards.  

\section{Breaking down tasks}
Starting big assignments and continuing to work on them consistently is a big struggle for me. Firstly I avoid starting tasks that I know or believe will take a long time and that will make me feel uncomfortable. In addition I find it hard to stick to tasks that I have started and tend to think to far ahead and it makes me quite stressed, for example I found myself thinking ahead of myself a lot with the dissertation but what I needed to focus on was reading the papers that I had found. I think I good way to solve this problem would be to break down tasks into lots of small and easily achievable tasks, for example instead of "Research learning using AR" I will find some papers and my tasks will be to read each one i.e. "Read paper X". This will make tasks much more manageable, and less daunting.    

\section*{Conclusion}
In conclusion I have identified five key skills that I would like to improve on. From doing this, I recognize that I have a lot of self development to do in order to be where I want to be in the future. I can become the person I want to be if I continue to reflect, plan, learn and evaluate my skills effectively. I will try hard to stick to my plans and goals, especially now that I am in my final year of this course and I would like to achieve a high final grade at the end of it. 

\bibliographystyle{ieeetran}
\bibliography{references}

\end{document}
